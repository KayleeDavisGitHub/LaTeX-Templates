%% Kyle Davis, Ohio State University

%  Using bibtex consider the following:
%! 1. you must have your .bib file and this working LaTeX document in the same location on your PC,
%     I reccomend a seperate folder, neatly organized for your project.
%
%! 2. Consider setting up a Quick Build shortcut:
%  Options -> Configure TexMaker -> Quick Build ->
%  User Wizard -> Set order:: Latex, bibtex, pdflatex, pdflatex(again),
%               pdfViewer. -> OK
%
% Then you can use quick build (up top) to generate updates and run your file!
%------------------------------ Preamble -----------------------------------------------
\documentclass[12pt]{article}
% Packages used. Explained in some more details at:
% https://github.com/KyleDavisGithub/LaTeX-Templates/blob/master/Memo-template.tex

\usepackage{scrextend}
\usepackage[T1]{fontenc}
% tex tools:
\usepackage{graphicx, fullpage, natbib, float, relsize, epsfig} 
% Math tools:
\usepackage{textcomp} % Just for tilda's I think
\usepackage{amsfonts, amsmath, amsthm, amssymb, mathtools} 
% Margins:
\usepackage[top = 1in, left = 1in, right = 1in, bottom = 1in]{geometry}
\usepackage{setspace}
\usepackage[bookmarks = false, hidelinks]{hyperref} 
\usepackage{booktabs}
\usepackage{array}
\usepackage{dcolumn}
\usepackage{float}
\usepackage{subcaption}
\usepackage{siunitx}
\usepackage[compact]{titlesec}
\usepackage{baskervald}
\usepackage{enumitem}
\usepackage[style = american]{csquotes}
\usepackage[american]{babel} 
\usepackage{verbatim} 
\usepackage{wrapfig}  % Wrap Figures and caption easier
\usepackage{caption}

%--------------------------- Formatting --------------------------------------------------------
%% For longer projects, consider to-do notes for workflow
% Very helpful for picking up where you left off!
\usepackage{todonotes}
% \todo 
% https://tex.stackexchange.com/questions/9796/how-to-add-todo-notes

% Footnote Formatting
\setlist{nolistsep, noitemsep}
\setlength{\footnotesep}{1.2\baselineskip}
\deffootnote[.2in]{0in}{0em}{\normalsize\thefootnotemark.\enskip}

% Section Formatting 
\def\ci{\perp\!\!\!\perp}
\titleformat*{\section}{\large\bfseries}
\titleformat*{\subsection}{\normalsize\bfseries}

% Table of Contents Formatting (babel) for renaming the Table of Contents (to "Navigation")
\addto\captionsamerican{ \renewcommand*\contentsname{Navigation} }


%------------------------------- Document ------------------------------------------------

\begin{document}
%\graphicspath{ path for images/figures } 
\begin{flushleft} 
\setlength{\parindent}{1cm} %1cm indent


% Doesn't apply page number
\thispagestyle{empty}     
\noindent \textbf{Introduction}\\ 
\hfill \\
Some Introduction before the table of contents. 

\hfill \\
% To put this on the next page:
% \clearpage 

% Table of Contents will read section titles and subtitles and list them alongside their associated page number.
\tableofcontents
% What's cool is that as your document grows you can simply click the TOC page to get to where you want to go!

%page style empty eliminates page count. The count I've set to begin later when it's page 1: onward.
\thispagestyle{empty}


%------------------------------- Main Body ------------------------------------------------

\clearpage
\setcounter{page}{1}

\section{A Section To Begin All Sections:}

Some text.

\section{Another:}

Follow up text.

\section{Yet Another:}

Yet more text.

\section{Nail Biting Conclusion:}

Ending text which will cite \citep{cary1967influence}.

\hfill \\

% To add a TOC item: 
\addcontentsline{toc}{section}{Optional Sub-Section}
\noindent \textbf{Optional Sub-Section to be Paired With Table of Contents Line:}

% Citing via .bib:
This text will cite things differently \cite{cary1967influence}. Note these citation codes are different, one just uses parentheses and the other will cite the authors last name along the year in parentheses. Added example \citep{waltz2001man}. Multiple authors \citep{hibbing2013predisposed}.





%-------------------------------Bibliography------------------------------------------------

\clearpage
% Groups bibtex cite options:
\begingroup
    % If commented out, this will not included references in
    % the references page if not \cite'ed in paper.
	\nocite{*}
	
	%\setlength{\bibsep}{12pt}
	
% To double space and stretch all cites:
	%\setstretch{2}
	
% Have your file in the same location as this tex document, then just name the .bib file here:
	\bibliography{BibFile}
	
% There's a ton of different options here, I prefer APSR but you can change this to anything!
%  It will code everything automatically.
	\bibliographystyle{apsr}
	% Or; apa
	%   ; plain
	%   ; apsr
	%   ; humannat
	%   ; chicago (?)
\endgroup

%-------------------------------Final Thoughts------------------------------------------------


% Consider keeping a .bib file for projects and papers for a few different reasons:

% 1. Changing formatting just a click away for different journals or picky reviewers.
% 2. All of your citations are in one place, a .bib can keep notes on each cite and their take aways (talk about concise note spaces!)
% 3. If colleagues ever want to learn more about a topic the .bib file can really help them get a quick head-start!

% Or consider keeping all of the articles you read in a .bib file where you can have notes on everything, and can easily cite when needed.


%---------------------------------------------------------------------------------------------
\end{flushleft}
\end{document}
